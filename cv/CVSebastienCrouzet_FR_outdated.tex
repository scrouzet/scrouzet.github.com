\documentclass[margin,line]{resume}

\usepackage[pdftex,                %
    bookmarks         = true,%     % Signets
    bookmarksnumbered = true,%     % Signets numerotes
    pdfpagemode       = None,%     % Signets/vignettes ferm� � l'ouverture
    pdfstartview      = FitH,%     % La page prend toute la largeur
    pdfpagelayout     = SinglePage, % Vue par page
    colorlinks        = true,%     % Liens en couleur
    % Pour le pdf
    linkcolor = blue, anchorcolor = blue, citecolor = green, menucolor = red, urlcolor = blue,
    % Pour l'impression
    %linkcolor = black, anchorcolor = black, citecolor = black, menucolor = black, urlcolor = magenta,
    pdfborder = {0 0 0}%   % Style de bordure : ici, pas de bordure
    ]{hyperref}%                   % Utilisation de HyperTeX

%Insertion des informations sur le document
\hypersetup{ % Modifiez la valeur des champs suivants
    pdfauthor   = {S\'ebastien M. Crouzet},%
    pdftitle    = {CV S\'ebastien M Crouzet}}


\begin{document}
\name{\Large S\'ebastien M. Crouzet}
\begin{resume}

     % Personal information
     \section{\mysidestyle Informations\\Personnelles}

    Age: 29\\
    Citoyennet\'e: Fran\c{c}ais\\
    Status marital : mari\'e, un enfant.
    \vspace{3mm}


    % Contact Information
    \section{\mysidestyle Contact}

    Thomas Serre Lab                            							\hfill phone: +1 401 400 1205 \\ 
    Department of Cognitive and Linguistic Sciences                          	\\%      \hfill fax: +33 5 62 17 28 09          \\
    Brown University										\hfill e-mail: \href{mailto:sebastien_crouzet@brown.edu}{sebastien\_crouzet@brown.edu} \\
    Providence, RI 02912, USA      							\hfill web: \url{http://www.brown.edu/Research/scrouzet/} 
    
    \vspace{3mm}
    % Current Position
    \section{\mysidestyle Emploi\\Actuel}
    Post-doctoral Associate\\ Department of Cognitive and Linguistic Sciences, Brown University, Providence, RI, USA
    
    \vspace{3mm}
    \section{\mysidestyle Emploi\\2012/2013}
    Post-doctoral Associate\\ 
    Charit\'e Universit\"atsmedizin Berlin, Institut f\"ur Medizinische Psychologie, Berlin, Germany\\
    Projet: Neuro-cognitive mechanisms of conscious and unconscious visual perception\\
    \url{http://www.uni-ulm.de/unbewusst/index.htm}\\
    
  %  \vspace{3mm}
    \section{\mysidestyle Projet}
En septembre 2012, je vais rentrer en Europe pour commencer un nouveau projet dans le laboratoire du Dr Niko Busch \`a Berlin, bas\'e sur l'\'electroenc\'ephalographie (EEG), afin d'\'etudier les processus visuels r\'e-entrants. Je travaille aussi actuellement \`a l'obtention d'un financement pour aller travailler avec Pieter Roelfsema \`a l'Universit\'e d'Amsterdam sur un projet impliquant des enregistrements intra-corticaux (iEEG) chez les patients \'epileptiques (probablement \`a partir de septembre 2013). Mon projet \`a moyen terme (2/3 ans) est ensuite de postuler aux concours du CNRS ou \`a des postes d'enseignant-chercheur en France.

\vspace{3mm}
    % Education
    \section{\mysidestyle \'Education}
    
    	\textbf{Doctorat de Neurosciences}, Universit\'e de Toulouse, CNRS, France \hfill \textbf{2010}\\
	Directeur :  Dr Simon J. Thorpe\\
	Sujet : Jeter un regard sur une phase pr\'ecoce des traitements visuels\\
	Mention tr\`{e}s honorable avec les f\'elicitations du jury \`{a} l'unanimit\'e.\\
	Date de la d\'efense : 12 juillet 2010
	\vspace{-1.5mm}
	
	\textbf{European Summer School in Visual Neurosciences} \hfill \textbf{Septembre 2008}\\
	\textsl{'From Spike to Awareness'}, Organisation: K. Gegenfurtner, F. Bremmer, J. Braun.\\
	Rauischholzhausen, Germany
	\vspace{-1.5mm}
	
	\textbf{Master de Sciences Cognitives}, \textsl{Mention bien} \hfill \textbf{2006}\\
	ENS / EHESS / Ecole Polytechnique / Paris 5 / Paris 6, France
	\vspace{-1.5mm}
	
	\textbf{Licence de Sciences Cognitives}, \textsl{Mention assez bien} \hfill \textbf{2004}\\
	Universit\'e Bordeaux 2, France


\vspace{3mm}	
% Publications
    \section{\mysidestyle Publications}

	\textbf{Articles dans des journaux \`a comit\'e de lecture}\\\\
	Crouzet SM and Thorpe SJ (2011). Low level cues and ultra-fast face detection. Front. Psychology 2:342. doi: 10.3389/fpsyg.2011.00342
	
	\vspace{-2mm} Crouzet SM and Serre T (2011). What are the visual features underlying rapid object recognition? Front. Psychology 2:326. doi: 10.3389/fpsyg.2011.00326
	
	\vspace{-2mm} Crouzet, S. M., Cauchoix, M. (2011). When does the visual system need to look back?  \textit{The Journal of Neuroscience}, 15 June 2011, 31(24): 8706-8707
	
\newpage
	
	\vspace{-2mm} Crouzet, S. M., Kirchner, H., \& Thorpe, S. J.  (2010). Fast saccades toward faces: Face detection in just 100 ms. \textit{Journal of Vision}, 10(4):16, 1-17, http://journalofvision.org/10/4/16/, doi:10.1167/10.4.16.
	
	
	\textbf{Chapitre de livre}\\\\
	M., Fabre-Thorpe, S. Crouzet, G. A. Rousselet, H. Kirchner and S. J. Thorpe (2008). Cat\'egorisation visuelle rapide: les visages sont-ils des 	objets sp\'ecifiques? In \textsl{Traitement et reconnaissance des visages: du percept \`a la personne}. E. J. Barbeau, S. Joubert and O. Felician. Marseille, Solal: 239-260.


\vspace{3mm}	
% Conference Presentations
    \section{\mysidestyle Pr\'esentations\\Conf\'erences}
    
Crouzet SM, Cauchoix M, Fize D, Serre T (2011) The neural basis of rapid categorization: Linking computational models and electrophysiology. NIPS 2011 workshop on machine learning and interpretation in neuroimaging.

Cauchoix M., Crouzet S., Fize D., Serre T. (2011) Visual features and dynamics of rapid recognition in monkey visual cortex. SFN 2011

Crouzet S M, Stemmler T, Capps M, Fahle M \& Serre T (2011) Single-trial decoding of binocular rivalry switches from oculometric and pupil data. Vision Science Society, Naples, Florida.

Brilhault A, Mathey M, Jolmes N, Crouzet S M \& Thorpe SJ (2011) Saccades to color: an ultra-fast controllable mechanism to low-level features. Vision Science Society, Naples, Florida.

Thorpe S J, Brilhault A, Mathey M, Crouzet S M, 2010, "Colour based target selection for ultrarapid saccades: The fastest controllable selection mechanism?" Perception 39 ECVP Abstract Supplement, page 158

Mathey M A, Crouzet S M, Thorpe S J, 2010, "The accuracy of ultra-rapid saccades to faces" Perception 39 ECVP Abstract Supplement, page 171

Crouzet, S. M. \& Thorpe, S. J. (2010) Power spectrum cues underlying ultra-fast saccades towards faces [Abstract]. Journal of Vision, 10(7): 634

Mathey, M. A., Crouzet, S. M. \& Thorpe, S. J. (2010) Ultra-rapid saccades to faces : the effect of target size [Abstract]. Journal of Vision, 10(7): 635

Crouzet S, Mathey M, Thorpe S J (2009). Ultra-fast saccades to faces: A temporal precedence effect? Perception 38 ECVP Abstract Supplement, page 157.

Crouzet, S. M., Joubert, O. R., Thorpe, S. J., \& Fabre-Thorpe, M. (2009). The bear before the forest, but the city before the cars: Revealing early object/background processing [Abstract]. Journal of Vision, 9(8):954

Fabre-Thorpe, M., Crouzet, S. M., Wu, C.-T., \& Thorpe, S. J. (2009). At 130 ms you "know" where the animal is but you don't yet "know" it's a dog [Abstract]. Journal of Vision, 9(8):786

Thorpe, S. J., Crouzet, S. M., Mac\'e, M. J., Bacon-Mac\'e, N., \& Fabre-Thorpe, M. (2009). Masking in a high-level gender discrimination task is essentially entirely pre-cortical [Abstract]. Journal of Vision, 9(8):546

S Crouzet, H Kirchner, S J Thorpe (2008). Saccading towards faces in 100 ms. What's the secret? Perception 37 ECVP Abstract Supplement, page 119. 

S J Thorpe, H Kirchner, S Crouzet, P Bayerl, H Neumann (2008). Processing times for optic flow patterns measured by the saccadic choice task. Perception 37 ECVP Abstract Supplement, page 40.
\newpage	
Crouzet, S., Thorpe, S. J., \& Kirchner, H. (2007). Category-dependent variations in visual processing time. Journal of Vision, 7(9):922,922a, http://journalofvision.org/7/9/922/, doi:10.1167/7.9.922.

Thorpe, S., Crouzet, S., \& Kirchner, H. (2007). Saliency maps and ultra-rapid choice saccade tasks. Journal of Vision, 7(9):30, 30a, http://journalofvision.org/7/9/30/, doi:10.1167/7.9.30.

Simon J. Thorpe, S\'ebastien Crouzet, Holle Kirchner and Mich\`{e}le Fabre-Thorpe (2006). Ultra-rapid face detection in natural images : implications for computation in the visual system. First French Conference on Computational Neurosciences, pp. 124-127. Abbaye des Pr\'emontr\'es, Pont à Mousson, France.

Simon J. Thorpe, S\'ebastien Crouzet and Holle Kirchner (2006). Comparing processing speed for complex natural scenes and simple visual forms. Perception, vol. 35, p 128.


	
% Invited talks
    \section{\mysidestyle Pr\'esentations\\Invit\'ees}
An early cortical basis for speed of sight (March 2012). David Sheinberg's lab meeting. Brown University, Providence, RI, USA

Rapid Visual Processing of Natural Scenes: Linking Behavioral and Electrophysiological Data to Computational Models (January 2012). CERCO-CNRS, Toulouse, France

Rapid Visual Processing of Natural Scenes: Linking Behavioral and Electrophysiological Data to Computational Models (November 2011). In-House Seminar, Neuroscience Department, Brown University, Providence, RI, USA

Revealing early visual processing of natural scenes using a saccade choice task (May 2009). Aude Oliva's lab meeting. MIT, Cambridge, MA, USA

\vspace{3mm}


   \section{\mysidestyle Responsibilit\'es\\ Professionnelles}
   
\textbf{Ad Hoc Reviewer}\\
\begin{tabular}{@{}p{6cm}p{6cm}}
Animal Cognition & Brain Topography\\
Frontiers in Perception Science & Seeing and Perceiving\\
\end{tabular}

\textbf{Review Editor}\\
Frontiers in Perception Science

\vspace{3mm}

% Teaching
    \section{\mysidestyle Enseignements}
	
	\textbf{Interventant invit\'e} \hfill \textbf{2011}\\
	\textsl{Computational Vision course, CLPS1520, Brown University, Providence, RI, USA}\\
	Object recognition in natural scenes
	
	\textbf{Charg\'e de Travaux Dirig\'es} \hfill \textbf{2006 \`a 2009}\\
	\textsl{UFR de Psychologie, Universit\'e Toulouse Le Mirail, Toulouse, France}\\
	Introduction aux Neurosciences
	
	\textbf{Charg\'e de cours} \hfill \textbf{2006 \`a 2009}\\
	\textsl{\'Ecole de Psychomotricit\'e, Facult\'e de M\'edecine de Rangueil, Toulouse, France}\\
	Le syst\`eme visuel 
	
	\textbf{Charg\'e de cours} \hfill \textbf{2006 \`a 2007}\\
	\textsl{\'Ecole de Psychomotricit\'e, Facult\'e de M\'edecine de Rangueil, Toulouse, France}\\
	Epistemologie de la neuropsychologie
	
	\textbf{Charg\'e de cours} \hfill \textbf{2006}\\
	\textsl{\'Ecole de Psychomotricit\'e, Facult\'e de M\'edecine de Rangueil, Toulouse, France}\\
	Sommeil, \'emotions

\newpage	


\vspace{3mm}
 % Official commitments
    \section{\mysidestyle Responsibilit\'es\\ Associations}

	\textbf{Co-organisateur du J3CN} \hfill \textbf{2010 \`a 2011}\\
	\textsl{Journal Club for Cognitive \& Computational Neuroscience, Brown University}\\
	 \url{https://sites.google.com/a/brown.edu/j3cn/}\\
	Providence, USA	

	\textbf{Principal organisateur du CJCSC'09} \hfill \textbf{2008 \`a 2009}\\
	\textsl{Colloque des Jeunes Chercheurs en Sciences Cognitives}\\
	 \url{http://fresco.risc.cnrs.fr/cjcsc2009/}\\
	Toulouse, France	


	\textbf{Principal organisateur de l'atelier PIRSTEC Jeunes Chercheurs} \hfill \textbf{2009}\\
	\textsl{Atelier de Prospective financ\'e par l'ANR}\\
	 \url{http://pirstec.risc.cnrs.fr}

	\textbf{Repr\'esentant non-staturaire au Conseil de laboratoire} \hfill \textbf{2006 to 2009}\\
	\textsl{Centre de Recherche Cerveau et Cognition}\\
	Toulouse, France	

	\textbf{Membre fondateur d'inCOGnu} \hfill \textbf{2006 to 2009}\\
	\textsl{Association des \'etudiants en sciences cognitives de Toulouse}\\
	 \url{http://incognu.fr/}\\
	Toulouse, France


\vspace{3mm}
% Fellowships
    \section{\mysidestyle Financements\\Bourses}

	%\textbf{Postgraduate grant} D\'el\'egation G\'en\'erale pour l'Armement (DGA, defense ministry) 2006/10 - 2009/09.\\
	%\textbf{Master grant} Ren\'e Descartes University (Paris 5) for the year 2005-2006.\\
	
	\textbf{Bourse de fin de th\`ese} \hfill \textbf{Nov 2009 \`a Mai 2010}\\
	\textsl{Fondation pour la Recherche M\'edicale (FRM)}
	
	\textbf{Bourse de th\`ese} \hfill \textbf{Oct 2006 \`a Sep 2009}\\ % ~ 100 000 Euros
	\textsl{D\'el\'egation G\'en\'erale pour l'Armement (DGA, Minist\`ere de la D\'efense)}
	
	\textbf{Bourse au m\'erite  de Master} \hfill \textbf{2005 to 2006}\\
	\textsl{Universit\'e Ren\'e Descartes (Paris 5)}
	

\end{resume}
\end{document}