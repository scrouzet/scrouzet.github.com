\documentclass[margin,line]{resume}

\usepackage[utf8]{inputenc}  
\usepackage[T1]{fontenc}
\usepackage{lmodern} % polices vectorielles Latin Modern (respecte + les conventions)
\usepackage[french]{babel}
%\usepackage{amsmath}
\usepackage[babel=true]{csquotes} % csquotes va utiliser la langue définie dans babel pour gerer les guillemets \enquote{}

%\usepackage{textcomp}% symboles supplementaires
%\usepackage{aeguill} % polices Adobe

\usepackage[pdftex,                %
    bookmarks         = true,%     % Signets
    bookmarksnumbered = true,%     % Signets numerotes
    pdfpagemode       = None,%     % Signets/vignettes fermé à l'ouverture
    pdfstartview      = FitH,%     % La page prend toute la largeur
    pdfpagelayout     = SinglePage, % Vue par page
    colorlinks        = true,%     % Liens en couleur
    % Pour le pdf
    linkcolor = blue, anchorcolor = blue, citecolor = green, menucolor = red, urlcolor = blue,
    % Pour l'impression
    %linkcolor = black, anchorcolor = black, citecolor = black, menucolor = black, urlcolor = magenta,
    pdfborder = {0 0 0}%   % Style de bordure : ici, pas de bordure
    ]{hyperref}%                   % Utilisation de HyperTeX

%Insertion des informations sur le document
\hypersetup{ % Modifiez la valeur des champs suivants
    pdfauthor   = {Sébastien M. Crouzet},%
    pdftitle    = {CV Sébastien M. Crouzet}}


\begin{document}
\name{\Large Sébastien M. Crouzet}
%\address{Berlin School of Mind and Brain, Luisenstraße 56, 10117 Berlin, Germany}
\begin{resume}

\section{\mysidestyle Informations\\Personnelles}
    Age: 31\\
    Citoyenneté: Français\\
    Statut marital : marié, un enfant.

\vspace{3mm}
\section{\mysidestyle Contact}
    Berlin School of Mind and Brain             \hfill Adresse électronique: \href{mailto:seb.crouzet@gmail.com}{seb.crouzet@gmail.com} \\
    Luisenstraße 56, 10117 Berlin, Germany      \hfill Page personnelle: \url{http://scrouzet.github.com} 

\vspace{3mm}
\section{\mysidestyle Mots clés}
    Perception; vision; neurosciences cognitives \& computationnelles; scènes naturelles; environnement réel; reconnaissance d'objets; \emph{machine learning}; dynamique de la perception.
    
\vspace{3mm}
\section{\mysidestyle \'Education \\ \& Expérience \\ Académique}
    
  \textbf{Chercheur post-doctoral}, Charité Universitätsmedizin Berlin, Allemagne \hfill \textbf{2012-2014}\\
	Superviseur: Niko Busch\\
	Sujet: Processus réentrants et conscience visuelle: mécanismes neuronaux et perceptifs.
	\vspace{-1.5mm}
	
	\textbf{Chercheur post-doctoral}, Brown University, Providence, RI, USA \hfill \textbf{2010-2012}\\
	Superviseur: Thomas Serre\\
	Sujet: Catégorisation de scènes naturelles: modèles computationnels \& électrophysiologie.
	\vspace{-1.5mm}
	
  \textbf{Doctorat de Neurosciences}, Université de Toulouse, CNRS, France \hfill \textbf{2010}\\
	Directeur :  Dr Simon J. Thorpe\\
	Sujet : Jeter un regard sur une phase précoce des traitements visuels\\
	Mention très honorable avec les félicitations du jury à l'unanimité.\\
	Date de la défense : 12 juillet 2010
	\vspace{-1.5mm}
	
	\textbf{Master de Sciences Cognitives}, \textsl{Mention bien} \hfill \textbf{2006}\\
	ENS / EHESS / Ecole Polytechnique / Paris 5 / Paris 6, France
	\vspace{-1.5mm}
	
	% \textbf{Licence de Sciences Cognitives}, \textsl{Mention assez bien} \hfill \textbf{2004}\\
	% Université Bordeaux 2, France
	% 
	% \textbf{DEUG de Psychologie} \hfill \textbf{2003}\\
	% Université Paris 5, France
	% \vspace{-1.5mm}
	% 
	% \textbf{Baccalauréat Scientifique, spécialité Mathématiques} \hfill \textbf{2000}\\
	% Lycée Bernard Palissy, Saintes, France


\vspace{3mm}
\section{\mysidestyle Formations \\ Additionnelles}

	\textbf{European Summer School in Visual Neurosciences} \hfill \textbf{Septembre 2008}\\
	\textsl{'From Spike to Awareness'}, Organisation: K. Gegenfurtner, F. Bremmer, J. Braun.\\
	Rauischholzhausen, Germany
	\vspace{-1.5mm}
	
	\textbf{Doctoriales de la DGA et de l'École Polytechnique} \hfill \textbf{Mai 2007}\\
	Fréjus, France
	\vspace{-1.5mm}


\vspace{3mm}	
\section{\mysidestyle Publications}

\textbf{Articles publiés dans des revues avec comité de lecture}\\\\
  \textbf{Crouzet SM}, Overgaard M \& Busch NA (2014). Visual masking leaves fastest saccadic responses intact. \textit{PLoS ONE}, 9(2), e87418. doi: 10.1371/journal.pone.0087418

	\vspace{-2mm} Cauchoix M and \textbf{Crouzet SM} (2013). How plausible is a subcortical account of rapid visual recognition?. \textit{Front. Hum. Neurosci.} 7:39. doi: 10.3389/fnhum.2013.00039
	
	\vspace{-2mm} \textbf{Crouzet SM}, Joubert OR, Thorpe SJ, Fabre-Thorpe M (2012) Animal Detection Precedes Access to Scene Category. \textit{PLoS ONE} 7(12): e51471. doi:10.1371/journal.pone.0051471

	\vspace{-2mm} \textbf{Crouzet SM} and Thorpe SJ (2011). Low level cues and ultra-fast face detection. \textit{Front. Psychology} 2:342. doi: 10.3389/fpsyg.2011.00342

	\vspace{-2mm} \textbf{Crouzet SM} and Serre T (2011). What are the visual features underlying rapid object recognition? \textit{Front. Psychology} 2:326. doi: 10.3389/fpsyg.2011.00326

	\vspace{-2mm} \textbf{Crouzet, S. M.}, Cauchoix, M. (2011). When does the visual system need to look back?  \textit{The Journal of Neuroscience}, 15 June 2011, 31(24): 8706-8707

	\vspace{-2mm} \textbf{Crouzet, S. M.}, Kirchner, H., \& Thorpe, S. J.  (2010). Fast saccades toward faces: Face detection in just 100 ms. \textit{Journal of Vision}, 10(4):16, 1-17, http://journalofvision.org/10/4/16/, doi:10.1167/10.4.16.


\newpage

\textbf{Articles à paraître dans des revues avec comité de lecture}\\\\ 
	Sofer, I., \& \textbf{Crouzet, S. M.}, \& Serre T. (en préparation). Principles of rapid visual scene categorization.

	\vspace{-2mm} Wu*, C.T., \textbf{Crouzet*, S. M.}, Thorpe, S.J. \& Fabre-Thorpe, M. (soumis). At 120 ms you can spot the animal but you don't yet know it's a dog. \textit{Soumis à Journal of Cognitive Neuroscience}.

	\vspace{-2mm} Cauchoix*, M., \textbf{Crouzet*, S. M.}, Fize, D., \& Serre T. (soumis). Fast ventral stream neural activity enables rapid visual categorization. \textit{Soumis à PNAS}.
	
 	\vspace{-2mm} * auteurs à contributions équivalentes 

\vspace{3mm}	
\textbf{Actes de congrès avec comité de lecture}\\\\
  \textbf{Crouzet SM}, Cauchoix M, Fize D, Serre T (2011) The neural basis of rapid categorization: Linking computational models and electrophysiology. NIPS 2011 workshop on machine learning and interpretation in neuroimaging. (Publication dans les actes retirée pour ne pas interfèrer avec la publication à venir dans un journal à comité de lecture.)

	\vspace{-2mm} Simon J. Thorpe, \textbf{Sébastien Crouzet}, Holle Kirchner and Michèle Fabre-Thorpe (2006). Ultra-rapid face detection in natural images : implications for computation in the visual system. First French Conference on Computational Neurosciences, pp. 124-127. Abbaye des Prémontrés, Pont à Mousson, France.


\vspace{3mm}	
\textbf{Chapitre de Livre}\\\\
	M., Fabre-Thorpe, \textbf{S. Crouzet}, G. A. Rousselet, H. Kirchner and S. J. Thorpe (2008). Catégorisation visuelle rapide: les visages sont-ils des 	objets spécifiques? In \textsl{Traitement et reconnaissance des visages: du percept à la personne}. E. J. Barbeau, S. Joubert and O. Felician. Marseille, Solal: 239-260.


\vspace{3mm}	
% Conference Presentations
    \section{\mysidestyle Présentations\\Conférences}

	\footnotesize % boring list so make it smaller

Sébastien M. Crouzet, Niko A. Busch, \& Kathrin Ohla (2014) Multivariate classification of ERP topographical information allows to investigate taste quality perception time-course. To be presented at Cutting EEG symposium in Berlin.

  \vspace{-2mm} Sébastien M. Crouzet, Simon Hviid Del Pin, Morten Overgaard \& Niko A. Busch (2014) Revealing the dynamics of visual masking using a speeded saccadic choice task. Submitted to VSS2014.

  \vspace{-2mm} Imri Sofer, Sébastien M. Crouzet \& Thomas Serre (2014) A simple rapid categorization model accounts for variations in behavioral responses across rapid scene categorization tasks. Submitted to VSS2014.
	
	\vspace{-2mm} Imri Sofer, Kwang Ryeol Lee, Pachaya Sailamul, Sébastien Crouzet, Thomas Serre (2013) Understanding the nature of the visual representations underlying rapid categorization tasks. [Abstract]. Journal of Vision, 13(9), article 658.

	\vspace{-2mm} Crouzet SM, Hviid Del Pin S, Overgaard M, Busch NA (2013) Dynamics of saccadic responses reveal how object substitution masking interferes with reentrant processing. 55th TeaP - Tagung experimentell arbeitender Psychologen (Conference of Experimental Psychologists).

	\vspace{-2mm} Crouzet SM, Cauchoix M, Fize D, Serre T (2011) The neural basis of rapid categorization: Linking computational models and electrophysiology. NIPS 2011 workshop on machine learning and interpretation in neuroimaging.

	\vspace{-2mm} Cauchoix M., Crouzet S., Fize D., Serre T. (2011) Visual features and dynamics of rapid recognition in monkey visual cortex. SFN 2011

	\vspace{-2mm} Crouzet S M, Stemmler T, Capps M, Fahle M \& Serre T (2011) Single-trial decoding of binocular rivalry switches from oculometric and pupil data. Vision Science Society, Naples, Florida.

	\vspace{-2mm} Brilhault A, Mathey M, Jolmes N, Crouzet S M \& Thorpe SJ (2011) Saccades to color: an ultra-fast controllable mechanism to low-level features. Vision Science Society, Naples, Florida.

	\vspace{-2mm} Thorpe S J, Brilhault A, Mathey M, Crouzet S M, 2010, "Colour based target selection for ultrarapid saccades: The fastest controllable selection mechanism?" Perception 39 ECVP Abstract Supplement, page 158

	\vspace{-2mm} Mathey M A, Crouzet S M, Thorpe S J, 2010, "The accuracy of ultra-rapid saccades to faces" Perception 39 ECVP Abstract Supplement, page 171

	\vspace{-2mm} Crouzet, S. M. \& Thorpe, S. J. (2010) Power spectrum cues underlying ultra-fast saccades towards faces [Abstract]. Journal of Vision, 10(7): 634

\newpage

	\vspace{-2mm} Mathey, M. A., Crouzet, S. M. \& Thorpe, S. J. (2010) Ultra-rapid saccades to faces : the effect of target size [Abstract]. Journal of Vision, 10(7): 635

	\vspace{-2mm} Crouzet S, Mathey M, Thorpe S J (2009). Ultra-fast saccades to faces: A temporal precedence effect? Perception 38 ECVP Abstract Supplement, page 157.

	\vspace{-2mm} Crouzet, S. M., Joubert, O. R., Thorpe, S. J., \& Fabre-Thorpe, M. (2009). The bear before the forest, but the city before the cars: Revealing early object/background processing [Abstract]. Journal of Vision, 9(8):954

	\vspace{-2mm} Fabre-Thorpe, M., Crouzet, S. M., Wu, C.-T., \& Thorpe, S. J. (2009). At 130 ms you "know" where the animal is but you don't yet "know" it's a dog [Abstract]. Journal of Vision, 9(8):786

	\vspace{-2mm} Thorpe, S. J., Crouzet, S. M., Macé, M. J., Bacon-Macé, N., \& Fabre-Thorpe, M. (2009). Masking in a high-level gender discrimination task is essentially entirely pre-cortical [Abstract]. Journal of Vision, 9(8):546

	\vspace{-2mm} S Crouzet, H Kirchner, S J Thorpe (2008). Saccading towards faces in 100 ms. What's the secret? Perception 37 ECVP Abstract Supplement, page 119. 
	
	\vspace{-2mm} S J Thorpe, H Kirchner, S Crouzet, P Bayerl, H Neumann (2008). Processing times for optic flow patterns measured by the saccadic choice task. Perception 37 ECVP Abstract Supplement, page 40.

	\vspace{-2mm} Crouzet, S., Thorpe, S. J., \& Kirchner, H. (2007). Category-dependent variations in visual processing time. Journal of Vision, 7(9):922,922a, http://journalofvision.org/7/9/922/, doi:10.1167/7.9.922.

	\vspace{-2mm} Thorpe, S., Crouzet, S., \& Kirchner, H. (2007). Saliency maps and ultra-rapid choice saccade tasks. Journal of Vision, 7(9):30, 30a, http://journalofvision.org/7/9/30/, doi:10.1167/7.9.30.

	\vspace{-2mm} Simon J. Thorpe, Sébastien Crouzet and Holle Kirchner (2006). Comparing processing speed for complex natural scenes and simple visual forms. Perception, vol. 35, p 128.

	\normalsize

\vspace{3mm}
\section{\mysidestyle Présentations\\Invitées}
    
	\footnotesize

	\textit{Invité par David Sheinberg}, Brown University, Providence, RI, USA \hfill \textbf{Mar 2012}\\
	An early cortical basis for speed of sight. 

	\vspace{-2mm} 
	\textit{Invité par Simon J. Thorpe}, CERCO-CNRS, Toulouse, France \hfill \textbf{Jan 2012}\\
	Rapid Visual Processing of Natural Scenes: Linking Behavioral and Electrophysiological Data to Computational Models.

	\vspace{-2mm} 
	\textit{In-House Seminar}, Neuroscience Department, Brown University, Providence, RI, USA \hfill \textbf{Nov 2011}\\
	Rapid Visual Processing of Natural Scenes: Linking Behavioral and Electrophysiological Data to Computational Models.

	\vspace{-2mm} 
	\textit{Invité par Aude Oliva}, MIT, Cambridge, MA, USA \hfill \textbf{May 2009}\\
	Revealing early visual processing of natural scenes using a saccade choice task.

	\normalsize

\vspace{3mm}
\section{\mysidestyle Qualifications\\MCF}
    \textbf{Qualification pour la fonction de Maître de conférences} - section 69 - Neurosciences\\
    08/02/2012 - 31/12/2016 (numéro de qualification : 12269224957)
    
    
\vspace{3mm}
\section{\mysidestyle Enseignements}
	
	\textbf{Chargé de cours} (14 sessions de 90 min) \hfill \textbf{2013/2014}\\
	\textsl{Programme de Master, Berlin School of Mind \& Brain, Berlin, Allemagne}\\
	Séminaire sur la perception visuelle. Enseignement en anglais.
	
	\vspace{-2mm}
	\textbf{Tutoriel à la Berlin School of Mind \& Brain} (6h) \hfill \textbf{Dec 2012}\\
	\textsl{Mind \& Brain Institute, Berlin, Germany}\\
	Utilisation de l'environnement R pour l'analyse de données, les statistiques et la visualisation.  Enseignement en anglais.

	\vspace{-2mm} 
	\textbf{Intervenant invité} (2h) \hfill \textbf{2011}\\
	\textsl{Computational Vision course, CLPS1520, Brown University, Providence, RI, USA}\\
	La reconnaissance d'objets dans les scènes naturelles. Enseignement en anglais.

	\vspace{-2mm} 
	\textbf{Chargé de Travaux Dirigés} (96h sur 3 ans) \hfill \textbf{2006 à 2009}\\
	\textsl{UFR de Psychologie, Université Toulouse Le Mirail, Toulouse, France}\\
	Introduction aux Neurosciences	

	\vspace{-2mm} 
	\textbf{Chargé de cours} (30h sur 3 ans) \hfill \textbf{2006 à 2009}\\
	\textsl{\'Ecole de Psychomotricité, Faculté de Médecine de Rangueil, Toulouse, France}\\
	Le système visuel 
	
\newpage
	
	\vspace{-2mm} 
	\textbf{Chargé de cours} (24h sur 2 ans) \hfill \textbf{2006 à 2007}\\
	\textsl{\'Ecole de Psychomotricité, Faculté de Médecine de Rangueil, Toulouse, France}\\
	\'Epistemologie de la neuropsychologie

	\vspace{-2mm} 
	\textbf{Chargé de cours} (10h) \hfill \textbf{2006}\\
	\textsl{\'Ecole de Psychomotricité, Faculté de Médecine de Rangueil, Toulouse, France}\\
	Sommeil, émotions


\vspace{3mm}
\section{\mysidestyle Supervision\\d'étudiants}
	
	\begin{tabular}{@{}ll} % @{} is to remove the white space left of the table
	Luca Lemi       & \textsl{\'Etudiant en thèse à la Berlin School of Mind \& Brain, Allemagne}\\
	Simon Ludwig    & \textsl{\'Etudiant en Master à la Freie Universität, Berlin, Allemagne}\\
	Maxime Cauchoix & \textsl{\'Etudiant en thèse à l'Université Toulouse 3 Paul Sabatier, Toulouse, France}\\
	Imri Sofer      & \textsl{\'Etudiant en thèse à Brown University, Providence, USA}\\
	Robin Martins   & \textsl{\'Etudiant Undergraduate à Brown University, Providence, USA} \\
	Rohan Katipally & \textsl{\'Etudiant Undergraduate à Brown University, Providence, USA} \\
	Marie Mathey    & \textsl{\'Etudiante en Master à Toulouse, France}
	\end{tabular}
	


%\vspace{3mm}
%\section{\mysidestyle Résumé de mes Activités de \\Vulgarisation Scientifique}
%Mes expériences avec l'enseignement m'ont permis de découvrir un exercice que j'apprécie particulièrement et qui m’a beaucoup apporté au niveau personnel, mais aussi dans mon travail de recherche. En plus de ces activités officielles, mon intérêt pour la transmission de savoirs scientifiques s’est manifesté à travers une activité associative importante. Plus précisément, j’ai été initiateur, puis membre fondateur, de l’association inCOGnu ayant pour but de mettre en relation les étudiants intéressés par la cognition venant de divers domaines sur la région toulousaine, mais aussi d’amener étudiants et chercheurs à présenter leurs travaux au grand public. Personnellement, je me suis aussi rendu chaque année de ma thèse dans des lycées de la région Toulousaine (Lycée Toulouse Lautrec, Lycée Toulouse Auzeville) afin d’initier un groupe d’élèves à l’étude de la perception visuelle et auditive, ou aux effets de la drogue sur le cerveau.


\vspace{3mm}
\section{\mysidestyle Services\\\'Editoriaux}
Animal Cognition; 
Attention, Perception, \& Psychophysics; 
Brain Topography;  
Cerebral Cortex; 
Frontiers in Perception Science (review editor); 
IEEE Transactions on Pattern Analysis and Machine Intelligence;
Journal of Vision;
PLoS ONE; 
Psychological Science;
Seeing and Perceiving.


\vspace{3mm}
 % Official commitments
    \section{\mysidestyle Responsabilités\\ Professionnelles\\ et Associatives}

	\textbf{Co-organisateur du J3CN} \hfill \textbf{2010 à 2011}\\
	\textsl{Journal Club for Cognitive \& Computational Neuroscience, Brown University}\\
	 \url{https://sites.google.com/a/brown.edu/j3cn/}\\
	Providence, USA	

	\vspace{-2mm} 
	\textbf{Organisateur du CJCSC'09} \hfill \textbf{2008 à 2009}\\
	\textsl{Colloque des Jeunes Chercheurs en Sciences Cognitives}\\
	Direction d'une équipe d'une vingtaine d'étudiants pour le comité d'organisation : recherche de financements, organisation scientifique et logistique du colloque.\\
	 \url{http://fresco.risc.cnrs.fr/cjcsc2009/}\\
	Toulouse, France	

	\vspace{-2mm} 
	\textbf{Organisateur de l'atelier PIRSTEC Jeunes Chercheurs} \hfill \textbf{2009}\\
	\textsl{Atelier de Prospective financé par l'ANR ayant eu lieu durant le CJCSC'09}\\
	\url{http://pirstec.risc.cnrs.fr}

	\vspace{-2mm} 
	\textbf{Représentant non-statutaire au Conseil de laboratoire} \hfill \textbf{2006 to 2009}\\
	\textsl{Centre de Recherche Cerveau et Cognition}\\
	Toulouse, France	

	\vspace{-2mm} 
	\textbf{Membre fondateur d'inCOGnu} \hfill \textbf{2006 to 2009}\\
	\textsl{Association des étudiants en sciences cognitives de Toulouse}\\
	 \url{http://incognu.fr/}\\
	Toulouse, France

	
\newpage

\vspace{3mm}
% Fellowships
    \section{\mysidestyle Financements /\\Bourses}

	\textbf{Financement via Niko Busch} \hfill \textbf{Depuis Septembre 2012}\\
	\textsl{Deutsche Forschungsgemeinschaft (DFG)}

	\vspace{-2mm} 
	\textbf{Financement via Thomas Serre} \hfill \textbf{Septembre 2010 à Juin 2012}\\
	\textsl{Defense Advanced Research Projects Agency (DARPA)}

	\vspace{-2mm} 
	\textbf{Financement via Simon J. Thorpe} \hfill \textbf{Mai 2010 à Juillet 2010}\\
	\textsl{Agence Nationale pour la Recherche (ANR)}
		
	\vspace{-2mm} 
	\textbf{Bourse de fin de thèse} \hfill \textbf{Novembre 2009 à Mai 2010}\\
	\textsl{Fondation pour la Recherche Médicale (FRM)}
		
	\vspace{-2mm} 
	\textbf{Bourse de thèse} \hfill \textbf{Octobre 2006 à Septembre 2009}\\ % ~ 100 000 Euros
	\textsl{Délégation Générale pour l'Armement (DGA, Ministère de la Défense)}
	
	\vspace{-2mm} 
	\textbf{Bourse au mérite  de Master} \hfill \textbf{2005 to 2006}\\
	\textsl{Université René Descartes (Paris 5)}	


\vspace{3mm}
\section{\mysidestyle Sociétés\\Professionnelles}
	Society for Neuroscience\\
	Vision Science Society
	
\vspace{3mm}
\section{\mysidestyle Langues}
	\textsl{Français}: Langue maternelle.\\
	\textsl{Anglais}: Courant.\\
	\textsl{Allemand}: En cours d'apprentissage.\\
	\textsl{Espagnol}: \'Elémentaire.


\vspace{3mm}
\section{\mysidestyle Compétences\\Techniques}
	\textsl{Systèmes d'exploitation}: Connaissances avancées des systèmes Mac OS X et GNU/Linux.\\
	\textsl{Langages de programmation}: MATLAB, R, Python.\\
	\textsl{Expériences}: Psychtoolbox pour MATLAB.\\
	\textsl{Oculomotricité}: SR Research Eyelink, SMI View Eyetracker, Chronos Eyetracker, EOG.\\
	\textsl{Enregistrements EEG}: BioSemi, Synamps.\\
	\textsl{Analyse EEG et iEEG}: Fonctions MATLAB faites maison + EEGlab.\\
	\textsl{Analyses Statistiques}: Tests paramétriques et non-paramétriques.\\
	\textsl{Machine learning}: Analyse multivariées (i.e. MVPA): classification et régression.\\
	\textsl{Communication et publications}: Connaissances avancées de \LaTeX, Adobe Illustrator / Inkscape \& Keynote (Mac OS); création de sites web avec HTML+CSS.


\vspace{3mm}
\section{\mysidestyle Références} 

	\begin{tabular}{@{}p{6cm}p{6cm}}
	\textbf{Dr Simon J. Thorpe}       &  \textbf{Dr Thomas Serre}                   \\
	Directeur de thèse                &  Superviseur de post-doc                       \\
	CNRS, Toulouse, France          &  Brown University, Providence, RI, USA        \\
	phone: \textsl{available on request}    &  phone: \textsl{available on request}     \\
	e-mail: \textsl{\href{mailto:simon.thorpe@cerco.ups-tlse.fr}{simon.thorpe@cerco.ups-tlse.fr}}   &  
	e-mail: \textsl{\href{mailto:thomas_serre@brown.edu}{thomas\_serre@brown.edu}}    \\
	\end{tabular}
	
	\textbf{Dr Niko A. Busch} \\
	Superviseur de post-doc\\
	Charité University, Berlin, Germany \\
	phone: \textsl{available on request} \\
	e-mail: \textsl{\href{mailto:niko.busch@charite.de}{niko.busch@charite.de}} \\


	
\end{resume}
\end{document}
