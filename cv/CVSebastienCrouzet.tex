\documentclass[margin,line]{resume}

\usepackage[utf8]{inputenc}  
\usepackage[T1]{fontenc}
\usepackage{lmodern} % polices vectorielles Latin Modern (respecte + les conventions)
%\usepackage{textcomp}% symboles supplementaires
%\usepackage{aeguill} % polices Adobe

\usepackage[bookmarks         = true,%     % Signets
    bookmarksnumbered = true,%     % Signets numerotes
    pdfpagemode       = None,%     % Signets/vignettes fermÔøΩ ÔøΩ l'ouverture
    pdfstartview      = FitH,%     % La page prend toute la largeur
    pdfpagelayout     = SinglePage, % Vue par page
    colorlinks        = true,%     % Liens en couleur
    % Pour le pdf
    linkcolor = blue, anchorcolor = blue, citecolor = green, menucolor = red, urlcolor = blue,
    % Pour l'impression
    %linkcolor = black, anchorcolor = black, citecolor = black, menucolor = black, urlcolor = magenta,
    pdfborder = {0 0 0}%   % Style de bordure : ici, pas de bordure
    ]{hyperref}%                   % Utilisation de HyperTeX

%Insertion des informations sur le document
\hypersetup{ % Modifiez la valeur des champs suivants
    pdfauthor   = {Sébastien M. Crouzet},%
    pdftitle    = {CV Sébastien M. Crouzet}}


\begin{document}
\name{\Large Sébastien M. Crouzet}
%\address{Berlin School of Mind and Brain, Luisenstraße 56, 10117 Berlin, Germany}
\begin{resume}

    % Personal information
    \section{\mysidestyle Civil\\Status}
    Age: 33\\
    Citizenship: French\\
    Married, two daughters.
    \vspace{3mm}
 

    % Contact Information
    \section{\mysidestyle Contact\\Information}
    CNRS CERCO UMR 5549             \hfill e-mail: \href{mailto:seb.crouzet@gmail.com}{seb.crouzet@gmail.com} \\
    Pavillon Baudot CHU Purpan 31052 Toulouse Cedex      \hfill web: \url{http://scrouzet.github.com} 
    % \hfill phone: +1 401 400 1205 \\ 
    % \hfill fax: +33 5 62 17 28 09 \\
    
	
	% --------------------------------------------------------------------------------------------------    
    \vspace{3mm}
    \section{\mysidestyle Current\\Position}
    \textbf{Post-doctoral Researcher}, Centre de Recherche Cerveau \& Cognition, Toulouse, France\\
    Principal Investigator: Rufin VanRullen\\
    Topic: Perceptual cycles and attention 

	% --------------------------------------------------------------------------------------------------
    \vspace{3mm}
    \section{\mysidestyle Education \\ \& Academic \\ Experience }
    
    	\textbf{Post-doctoral Researcher}, Charité University, Berlin, Germany \hfill \textbf{2012-2014}\\
    	Principal Investigator: Niko Busch\\
    	Topic: Reentrant processing and visual awareness: neural and perceptual mechanisms 
	\vspace{-1.5mm}
	
	\textbf{Post-doctoral Researcher}, Brown University, Providence, RI, USA \hfill \textbf{2010-2012}\\
	Principal Investigator: Thomas Serre\\
	Topic: Linking behavioral and electrophysiological data to computational models
	\vspace{-1.5mm}
	
    \textbf{Ph.D. in Neurosciences}, Université de Toulouse, CNRS, France \hfill \textbf{2010}\\
	Advisor:  Dr Simon J. Thorpe\\
	Committee: F. Vitu-Thibault, G.A. Rousselet, O. Pascalis, B. Rossion, P-G Zanone, D. Bazalgette\\
	Topic: Ultra-rapid recognition of objects in natural scenes.\\
	Highest academic distinction: Très honorable avec les félicitations du jury à l'unanimité.
	\vspace{-1.5mm}

	\textbf{European Summer School in Visual Neurosciences} \hfill \textbf{Sep 2008}\\
	\textsl{'From Spike to Awareness'}, Organisation: K. Gegenfurtner, F. Bremmer, J. Braun.\\
	Rauischholzhausen, Germany
	\vspace{-1.5mm}

	\textbf{Master in Cognitive Science}, \textsl{Graduated magna cum laude} \hfill \textbf{2006}\\
	ENS / EHESS / Ecole Polytechnique / Paris 5 / Paris 6, France
	\vspace{-1.5mm}

	\textbf{Licence in Cognitive Science}, \textsl{Graduated magna cum laude} \hfill \textbf{2004}\\
	Université Bordeaux 2, France
	\vspace{-1.5mm}
	
	\textbf{DEUG in Psychology} \hfill \textbf{2003}\\
	Université Paris 5, France
	\vspace{-1.5mm}
	
	\textbf{Baccalauréat Scientifique, spécialité Mathématiques} \hfill \textbf{2000}\\
	Lycée Bernard Palissy, Saintes, France

	% --------------------------------------------------------------------------------------------------
	%	Sofer, I., \& \textbf{Crouzet, S. M.}, \& Serre T. (in preparation). Principles of rapid visual scene categorization.

%	\vspace{-2mm} Cauchoix*, M., \textbf{Crouzet*, S. M.}, Fize, D., \& Serre T. (in revision). Fast ventral stream neural activity enables rapid visual categorization. \textit{PNAS}.

	\vspace{3mm}	
    \section{\mysidestyle Refereed Journal Articles}
    	  Cauchoix, M., \textbf{Crouzet, S. M.}, Fize, D., \& Serre, T. (2016). Fast ventral stream neural activity enables rapid visual categorization. \textit{NeuroImage}, 125, 280–290. doi:10.1016/j.neuroimage.2015.10.012

	\vspace{-2mm} Sofer, I., \textbf{Crouzet, S. M.}, \& Serre, T. (2015) Explaining the Timing of Natural Scene Understanding with a Computational Model of Perceptual Categorization. \textit{PLoS Computational Biology}, 11(9): e1004456.

	\vspace{-2mm} Chaumon, M., \textbf{Crouzet, S. M.}, \& Busch N.A. (2015). Cutting-edge methods for EEG research on cognition. \textit{Journal of Neuroscience Methods}, 250, 1-2.

	\vspace{-2mm}  \textbf{Crouzet*, S. M.}, Busch*, N.A. \& Ohla, K. (2015). Taste quality decoding parallels taste sensations.\textit{Current Biology}, 25, 1-7.
		
	\vspace{-2mm}  Wu*, C.T.,  \textbf{Crouzet*, S. M.}, Thorpe, S.J. \& Fabre-Thorpe, M. (2015). At 120 ms you can spot the animal but you don't yet know it's a dog. \textit{Journal of Cognitive Neuroscience},  27(1) : 141-149.
			
  	\vspace{-2mm}  \textbf{Crouzet SM}, Overgaard M \& Busch NA (2014). Visual masking leaves fastest saccadic responses intact. \textit{PLoS ONE}, 9(2), e87418.

  	\newpage
	
	\vspace{-2mm} Cauchoix M and \textbf{Crouzet SM} (2013). How plausible is a subcortical account of rapid visual recognition?. Front. Hum. Neurosci. 7:39.
	
	\vspace{-2mm} \textbf{Crouzet SM}, Joubert OR, Thorpe SJ, Fabre-Thorpe M (2012) Animal Detection Precedes Access to Scene Category. PLoS ONE 7(12): e51471.

	\vspace{-2mm} \textbf{Crouzet SM} and Thorpe SJ (2011). Low level cues and ultra-fast face detection. Front. Psychology 2:342.

	\vspace{-2mm} \textbf{Crouzet SM} and Serre T (2011). What are the visual features underlying rapid object recognition? Front. Psychology 2:326.

	\vspace{-2mm} \textbf{Crouzet, S. M.}, Cauchoix, M. (2011). When does the visual system need to look back?  \textit{The Journal of Neuroscience}, 15 June 2011, 31(24): 8706-8707.

	\vspace{-2mm} \textbf{Crouzet, S. M.}, Kirchner, H., \& Thorpe, S. J.  (2010). Fast saccades toward faces: Face detection in just 100 ms. \textit{Journal of Vision}, 10(4):16, 1-17.
		
%	\vspace{-2mm} * auteurs à contributions équivalentes 

%\newpage

	% --------------------------------------------------------------------------------------------------
	\vspace{3mm}	
    \section{\mysidestyle Book Chapter}

M., Fabre-Thorpe, \textbf{S. Crouzet}, G. A. Rousselet, H. Kirchner and S. J. Thorpe (2008). Catégorisation visuelle rapide: les visages sont-ils des objets spécifiques? In \textsl{Traitement et reconnaissance des visages: du percept à la personne}. E. J. Barbeau, S. Joubert and O. Felician. Marseille, Solal: 239-260.

	% --------------------------------------------------------------------------------------------------
	\vspace{3mm}	
    \section{\mysidestyle Conference\\Presentations}

\footnotesize % boring list so make it smaller

Kathrin Ohla, Niko A. Busch \& Sébastien M. Crouzet (2016) Dynamic coding of taste categories in the human brain. ISOT 2016 (17th International Symposium on Olfaction and Taste).

\vspace{-2mm} Luca Iemi, Lena Walther, Sébastien Crouzet, Maximilien Chaumon \& Niko A. Busch (2015) Uncovering the role of spontaneous alpha oscillations in visual conscious processing. ASSC19 in Paris (Association for the Scientific Study of Consciousness).
	
\vspace{-2mm} Kathrin Ohla, Niko A. Busch \& Sébastien M. Crouzet (2014) Decoding neural taste quality processing with multivariate pattern analyses (MVPA) of human brain electric activity. AChemS (Association for Chemoreception Sciences) 36th Annual Meeting.

\vspace{-2mm} Sébastien M. Crouzet, Simon Hviid Del Pin, Morten Overgaard \& Niko A. Busch (2014) Revealing the dynamics of visual masking using a speeded saccadic choice task. VSS2014.

\vspace{-2mm} Imri Sofer, Sébastien M. Crouzet \& Thomas Serre (2014) A simple rapid categorization model accounts for variations in behavioral responses across rapid scene categorization tasks. VSS2014.

\vspace{-2mm} Sébastien M. Crouzet, Niko A. Busch, \& Kathrin Ohla (2014) Multivariate classification of ERP topographical information allows to investigate taste quality perception time-course. Cutting EEG 2014.

\vspace{-2mm} Imri Sofer, Kwang Ryeol Lee, Pachaya Sailamul, Sébastien Crouzet \& Thomas Serre (2013) Understanding the nature of the visual representations underlying rapid categorization tasks.  [Abstract]. Journal of Vision, 13(9), article 658.

\vspace{-2mm} Crouzet SM, Hviid Del Pin S, Overgaard M \& Busch NA (2013) Dynamics of saccadic responses reveal how object substitution masking interferes with reentrant processing. 55th TeaP - Tagung experimentell arbeitender Psychologen (Conference of Experimental Psychologists).

\vspace{-2mm} Crouzet SM, Cauchoix M, Fize D \& Serre T (2011) The neural basis of rapid categorization: Linking computational models and electrophysiology. NIPS 2011 workshop on machine learning and interpretation in neuroimaging.

\vspace{-2mm} Cauchoix M., Crouzet S., Fize D. \& Serre T. (2011) Visual features and dynamics of rapid recognition in monkey visual cortex. SFN 2011

\vspace{-2mm} Crouzet S M, Stemmler T, Capps M, Fahle M \& Serre T (2011) Single-trial decoding of binocular rivalry switches from oculometric and pupil data. Vision Science Society, Naples, Florida.

\vspace{-2mm} Brilhault A, Mathey M, Jolmes N, Crouzet S M \& Thorpe SJ (2011) Saccades to color: an ultra-fast controllable mechanism to low-level features. Vision Science Society, Naples, Florida.

\vspace{-2mm} Thorpe S J, Brilhault A, Mathey M, Crouzet S M, 2010, "Colour based target selection for ultrarapid saccades: The fastest controllable selection mechanism?" Perception 39 ECVP Abstract Supplement, page 158

\vspace{-2mm} Mathey M A, Crouzet S M, Thorpe S J, 2010, "The accuracy of ultra-rapid saccades to faces" Perception 39 ECVP Abstract Supplement, page 171

\vspace{-2mm} Crouzet, S. M. \& Thorpe, S. J. (2010) Power spectrum cues underlying ultra-fast saccades towards faces [Abstract]. Journal of Vision, 10(7): 634

\vspace{-2mm} Mathey, M. A., Crouzet, S. M. \& Thorpe, S. J. (2010) Ultra-rapid saccades to faces : the effect of target size [Abstract]. Journal of Vision, 10(7): 635

\vspace{-2mm} Crouzet S, Mathey M, Thorpe S J (2009). Ultra-fast saccades to faces: A temporal precedence effect? Perception 38 ECVP Abstract Supplement, page 157.

\vspace{-2mm} Crouzet, S. M., Joubert, O. R., Thorpe, S. J., \& Fabre-Thorpe, M. (2009). The bear before the forest, but the city before the cars: Revealing early object/background processing [Abstract]. Journal of Vision, 9(8):954

\newpage

\vspace{-2mm} Fabre-Thorpe, M., Crouzet, S. M., Wu, C.-T., \& Thorpe, S. J. (2009). At 130 ms you "know" where the animal is but you don't yet "know" it's a dog [Abstract]. Journal of Vision, 9(8):786

\vspace{-2mm} Thorpe, S. J., Crouzet, S. M., Macé, M. J., Bacon-Macé, N., \& Fabre-Thorpe, M. (2009). Masking in a high-level gender discrimination task is essentially entirely pre-cortical [Abstract]. Journal of Vision, 9(8):546

\vspace{-2mm} S Crouzet, H Kirchner, S J Thorpe (2008). Saccading towards faces in 100 ms. What's the secret? Perception 37 ECVP Abstract Supplement, page 119. 

\vspace{-2mm} S J Thorpe, H Kirchner, S Crouzet, P Bayerl, H Neumann (2008). Processing times for optic flow patterns measured by the saccadic choice task. Perception 37 ECVP Abstract Supplement, page 40.

\vspace{-2mm} Crouzet, S., Thorpe, S. J., \& Kirchner, H. (2007). Category-dependent variations in visual processing time. Journal of Vision, 7(9):922,922a, http://journalofvision.org/7/9/922/, doi:10.1167/7.9.922.

\vspace{-2mm} Thorpe, S., Crouzet, S., \& Kirchner, H. (2007). Saliency maps and ultra-rapid choice saccade tasks. Journal of Vision, 7(9):30, 30a, http://journalofvision.org/7/9/30/, doi:10.1167/7.9.30.

\vspace{-2mm} Simon J. Thorpe, Sébastien Crouzet, Holle Kirchner and Michèle Fabre-Thorpe (2006). Ultra-rapid face detection in natural images : implications for computation in the visual system. First French Conference on Computational Neurosciences, pp. 124-127. Abbaye des Prémontrés, Pont à Mousson, France.

\vspace{-2mm} Simon J. Thorpe, Sébastien Crouzet and Holle Kirchner (2006). Comparing processing speed for complex natural scenes and simple visual forms. Perception, vol. 35, p 128.

\normalsize


	% --------------------------------------------------------------------------------------------------
    \section{\mysidestyle Invited\\Talks}
    
\footnotesize % boring list so make it smaller

\textit{Invited by Christophe Jouffrais}, IRIT (Toulouse Computer Science Research Institute), France \hfill \textbf{Oct 2015}\\
Studying the dynamics of perception using machine learning.

\textit{Cutting EEG 2015}, Berlin School of Mind \& Brain, Germany \hfill \textbf{29 September - 2 October 2015}\\
Time-resolved MVPA for EEG analysis. \\
\url{http://www.mind-and-brain.de/cutting-eeg-2015}
	
\textit{Invited by David Sheinberg}, Brown University, Providence, RI, USA \hfill \textbf{Mar 2012}\\
An early cortical basis for speed of sight. 

\vspace{-2mm} 
\textit{Invited by Simon J. Thorpe}, CERCO-CNRS, Toulouse, France \hfill \textbf{Jan 2012}\\
Rapid Visual Processing of Natural Scenes: Linking Behavioral and Electrophysiological Data to Computational Models.

\vspace{-2mm} 
\textit{In-House Seminar}, Neuroscience Department, Brown University, Providence, RI, USA \hfill \textbf{Nov 2011}\\
Rapid Visual Processing of Natural Scenes: Linking Behavioral and Electrophysiological Data to Computational Models.

\vspace{-2mm} 
\textit{Invited by Aude Oliva}, MIT, Cambridge, MA, USA \hfill \textbf{May 2009}\\
Revealing early visual processing of natural scenes using a saccade choice task.

\normalsize


	% --------------------------------------------------------------------------------------------------
	\vspace{3mm}
	\section{\mysidestyle Editorial\\Service}

 \textbf{Reviewing and editing for scientific journals} \\
Animal Cognition; 
Attention, Perception, \& Psychophysics; 
Brain Topography;  
Cerebral Cortex; 
Cognition and Emotion;
European Journal of Neuroscience;
Frontiers in Perception Science (review editor); 
Frontiers in Human Neuroscience (review editor); 
Frontiers in Computational Neuroscience; 
IEEE Transactions on Pattern Analysis and Machine Intelligence;
Journal of Experimental Psychology: Human Perception and Performance;
Journal of Neuroscience;
Journal of Neuroscience Methods;
Journal of Vision;
Perception;
PLoS ONE; 
Psychological Science;
Robotics and Autonomous Systems;
Seeing and Perceiving;
Vision Research.

\textbf{Program committee for conferences} \\
Member of the program committee of the The First International Workshop on Computational Models of the Visual Cortex (CMVC): Hierarchies, Layers, Sparsity, Saliency and Attention. Held as part of the Bio-inspired Information and Communications Technologies conference (BICT), New York, 3-5 December, 2015.
 \url{http://cmvc.bionetics.org/2015/show/home}
 
	% --------------------------------------------------------------------------------------------------
	\vspace{3mm}
	\section{\mysidestyle Teaching\\Certification}
    \textbf{Qualification pour la fonction de Maître de conférences} - section 69 - Neurosciences\\
    08/02/2012 - 31/12/2016 (numéro de qualification : 12269224957)


\newpage

	% --------------------------------------------------------------------------------------------------       
	\vspace{3mm}
    \section{\mysidestyle Teaching\\Experience}
    
   	 \textbf{Instructor} (14 sessions of 90 min) \hfill \textbf{2013/2014}\\
	\textsl{Master program, Berlin School of Mind \& Brain, Berlin, Germany}\\
	Seminar on visual perception. Teaching in English language.
	
    	\textbf{Statistics Tutorial} (6h) \hfill \textbf{Dec 2012}\\
	\textsl{Doctoral school, Berlin School of Mind \& Brain, Berlin, Germany}\\
	Using the R environment for data analysis, statistical computing and graphics. Teaching in English language.

	\textbf{Guest lecture} (2h) \hfill \textbf{2011}\\
	\textsl{Computational Vision course, CLPS1520, Brown University, Providence, RI, USA}\\
	Object recognition in natural scenes. Teaching in English language.

	\textbf{Teaching Assistant} (96h over 3 years) \hfill \textbf{2006 to 2009}\\
	\textsl{Department of Psychology, Université Toulouse Le Mirail, Toulouse, France}\\
	Introduction to Neurosciences

	\textbf{Instructor} (30h over 3 years) \hfill \textbf{2006 to 2009}\\
	\textsl{School of Psychomotricity, Faculté de Médecine de Rangueil, Toulouse, France}\\
	Visual system and eye movements

	\textbf{Instructor} (24h over 2 years) \hfill \textbf{2006 to 2007}\\
	\textsl{School of Psychomotricity, Faculté de Médecine de Rangueil, Toulouse, France}\\
	Epistemology of neuropsychology



	\textbf{Instructor} (10h) \hfill \textbf{2006}\\
	\textsl{School of Psychomotricity, Faculté de Médecine de Rangueil, Toulouse, France}\\
	Sleep, emotions


	% --------------------------------------------------------------------------------------------------
	\vspace{3mm}
    \section{\mysidestyle Academic\\Mentoring}
	
	\textbf{As a PhD student and post-doc, I have worked with:}\\
	\begin{tabular}{@{}ll} % @{} is to remove the white space left of the table
	Luca Iemi       & \textsl{PhD student at the Berlin School of Mind and Brain, Germany}\\
	Simon Ludwig    & \textsl{Master student at Freie Universität, Berlin, Germany}\\
	Maxime Cauchoix & \textsl{PhD student at Université Toulouse 3 Paul Sabatier, Toulouse, France}\\
	Imri Sofer      & \textsl{PhD student at Brown University, Providence, USA}\\
	Robin Martins   & \textsl{Undergraduate student at Brown University, Providence, USA} \\
	Rohan Katipally & \textsl{Undergraduate student at Brown University, Providence, USA} \\
	Marie Mathey    & \textsl{Master student in Toulouse, France}
	\end{tabular}
	


	% --------------------------------------------------------------------------------------------------
	\vspace{3mm}
    \section{\mysidestyle Official\\ Commitments} % ou Responsibilities/Associations
	
	\textbf{Organizer of Cutting EEG 2014} \hfill \textbf{19–21 February 2014}\\
	\textsl{Member of the organizing committee for the Cutting EEG 2014: Symposium on cutting-edge EEG methods. Specifically in charge of the proceedings' publication in Journal of Neuroscience Methods.}\\
	 \url{http://www.mind-and-brain.de/postdoctoral-program/scientific-events/cutting-eeg/}\\
	Berlin, Germany
	
	\textbf{Organizer of the J3CN} \hfill \textbf{2010 to 2011}\\
	\textsl{Journal Club for Cognitive \& Computational Neuroscience, Brown University}\\
	 \url{https://sites.google.com/a/brown.edu/j3cn/}\\
	Providence, USA	

	\textbf{Header of the Organizing Committee for the CJCSC'09} \hfill \textbf{2008 to 2009}\\
	\textsl{French Cognitive Science Young Researcher Conference}\\
	 \url{http://fresco.risc.cnrs.fr/cjcsc2009/}\\
	Toulouse, France		

	\textbf{Header of the Young Researcher Workshop for trend forecasting} \hfill \textbf{2009}\\
	 \textbf{in Cognitive Science}\\
	\textsl{Part of the PIRSTEC project funded by the French National Research Agency (ANR)}\\
	 \url{http://pirstec.risc.cnrs.fr}
	 
	 	\newpage

	\textbf{Students and Post-Docs representative} \hfill \textbf{2006 to 2009}\\
	\textsl{Brain and Cognition Research Center lab council}\\
	Toulouse, France	

	\textbf{Founding member of the association inCOGnu} \hfill \textbf{2006 to 2009}\\
	\textsl{Association of cognitive science students of Toulouse}\\
	 \url{http://incognu.fr/}\\
	Toulouse, France


	% --------------------------------------------------------------------------------------------------
	\vspace{3mm}
	\section{\mysidestyle Fellowships, Grants \& Fundings}

	\textbf{ERC grant awarded to Rufin VanRullen} \hfill \textbf{From Sep 2014}\\
	\textsl{European Research Counsil Consolidator Grant: P-cycles}

	\textbf{Grant awarded to Niko Busch} \hfill \textbf{Sep 2012 to Aug 2014}\\
	\textsl{Deutsche Forschungsgemeinschaft (DFG)}

	\textbf{Grant awarded to Thomas Serre} \hfill \textbf{Sep 2010 to Jul 2012}\\
	\textsl{Defense Advanced Research Projects Agency (DARPA).\\ I had an active participation in the monthly+trimestrial+annual reports.}
	
	\textbf{4th year of Ph.D. fellowship} \hfill \textbf{Nov 2009 to May 2010}\\
	\textsl{Fondation pour la Recherche Médicale (FRM)}	
	
	\textbf{Postgraduate scholarship} \hfill \textbf{Oct 2006 to Sep 2009}\\ % ~ 100 000 Euros
	\textsl{Délégation Générale pour l'Armement (DGA, French Ministry of Defense)}

	\textbf{Master scholarship (bourse d’excellence)} \hfill \textbf{Sep 2005 to Jun 2006}\\
	\textsl{Université René Descartes Paris 5}


	% --------------------------------------------------------------------------------------------------
%\vspace{3mm}
%\section{\mysidestyle Awards \& Scientific\\ Prizes}
%
%	\textbf{Preselected for the DGA 2010 thesis prize}\\
%	Ministère de la Défense et des Anciens Combattants et le Ministère de la Recherche.
%
%	\textbf{Travel award for the European Summer School in Visual Neurosciences}\\
%	“From Spike to Awareness” (September 2008). Organisation: K. Gegenfurtner, F. Bremmer, J. Braun. Rauischholzhausen, Germany.
%
%	\textbf{Selected in the top 5 posters at the DGA Doctoriales 2008}

	
	% --------------------------------------------------------------------------------------------------
	\vspace{3mm}
	\section{\mysidestyle Professional\\Societies}
	Society for Neuroscience\\
	Vision Science Society
	

	% --------------------------------------------------------------------------------------------------
%	\vspace{3mm}
%    \section{\mysidestyle Languages}
%	\textsl{French}: Mother tongue\\
%	\textsl{English}: Fluent\\
%	\textsl{German}: Trying to learn\\
%	\textsl{Spanish}: Very elementary

%\vspace{3mm}


	% --------------------------------------------------------------------------------------------------
	\vspace{3mm}
    \section{\mysidestyle Professional\\Skills}
	\textsl{Operating Systems}: Advanced knowledge of Mac OS and GNU Linux.\\
	\textsl{Programming languages}: MATLAB, R, Python.\\
	\textsl{Experimental testing}: Psychtoolbox for MATLAB.\\
	\textsl{Eye movement recording}: SR Research Eyelink, SMI View Eyetracker, Chronos Eyetracker, EOG.\\
	\textsl{EEG and iEEG analysis}: Homemade MATLAB functions and EEGlab.\\
	\textsl{Statistical Analysis}: Parametric and non-parametric tests, Multivariate Pattern Analysis.\\
	\textsl{Communication and publishing}: Advanced knowledge of \LaTeX, Adobe Illustrator \& the presentation software Keynote (Mac OS); website creation and maintenance with HTML and CSS.


	% --------------------------------------------------------------------------------------------------
	\section{\mysidestyle Referees} 

	\begin{tabular}{@{}p{6cm}p{6cm}}
	\textbf{Dr Simon J. Thorpe}       &  \textbf{Dr Thomas Serre}                   \\
	Ph.D. advisor                               &  Post-doc advisor                       \\
	CNRS, Toulouse, France          &  Brown University, Providence, RI, USA        \\
	phone: \textsl{available on request}    &  phone: \textsl{available on request}     \\
	\textsl{\href{mailto:simon.thorpe@cerco.ups-tlse.fr}{simon.thorpe@cerco.ups-tlse.fr}}   &  
	\textsl{\href{mailto:thomas_serre@brown.edu}{thomas\_serre@brown.edu}}    \\
	\end{tabular}
	
	\begin{tabular}{@{}p{6cm}p{6cm}}
	\textbf{Dr Niko A. Busch}       &  \textbf{Dr Rufin VanRullen}                   \\
	Post-doc advisor                   &  Post-doc advisor                       \\
	Charité University, Berlin, Germany         &  CNRS, Toulouse, France        \\
	phone: \textsl{available on request}    &  phone: \textsl{available on request}     \\
	\textsl{\href{mailto:niko.busch@charite.de}{niko.busch@charite.de}}  &  
	\textsl{\href{mailto:rufin.vanrullen@cerco.ups-tlse.fr}{rufin.vanrullen@cerco.ups-tlse.fr}}    \\
	\end{tabular}


\end{resume}
\end{document}
