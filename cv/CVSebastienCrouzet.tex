\documentclass[margin,line]{resume}

\usepackage[utf8]{inputenc}  
\usepackage[T1]{fontenc}
\usepackage{lmodern} % polices vectorielles Latin Modern (respecte + les conventions)
%\usepackage{textcomp}% symboles supplementaires
%\usepackage{aeguill} % polices Adobe

\usepackage[pdftex,                %
    bookmarks         = true,%     % Signets
    bookmarksnumbered = true,%     % Signets numerotes
    pdfpagemode       = None,%     % Signets/vignettes ferm� � l'ouverture
    pdfstartview      = FitH,%     % La page prend toute la largeur
    pdfpagelayout     = SinglePage, % Vue par page
    colorlinks        = true,%     % Liens en couleur
    % Pour le pdf
    linkcolor = blue, anchorcolor = blue, citecolor = green, menucolor = red, urlcolor = blue,
    % Pour l'impression
    %linkcolor = black, anchorcolor = black, citecolor = black, menucolor = black, urlcolor = magenta,
    pdfborder = {0 0 0}%   % Style de bordure : ici, pas de bordure
    ]{hyperref}%                   % Utilisation de HyperTeX

%Insertion des informations sur le document
\hypersetup{ % Modifiez la valeur des champs suivants
    pdfauthor   = {Sébastien M. Crouzet},%
    pdftitle    = {CV Sébastien M. Crouzet}}


\begin{document}
\name{\Large Sébastien M. Crouzet}
%\address{Berlin School of Mind and Brain, Luisenstraße 56, 10117 Berlin, Germany}
\begin{resume}

    % Personal information
    \section{\mysidestyle Personal\\Information}
    Age: 30\\
    Citizenship: French\\
    Marital status: married, one daughter.
    \vspace{3mm}


    % Contact Information
    \section{\mysidestyle Contact\\Information}
    Berlin School of Mind and Brain             \hfill e-mail: \href{mailto:seb.crouzet@gmail.com}{seb.crouzet@gmail.com} \\
    Luisenstraße 56, 10117 Berlin, Germany      \hfill web: \url{http://scrouzet.github.com} 
    % \hfill phone: +1 401 400 1205 \\ 
    % \hfill fax: +33 5 62 17 28 09 \\
    
    
    \vspace{3mm}
    % Current Position
    \section{\mysidestyle Current\\Position}
    \textbf{Post-doctoral Associate} in Niko Busch’s laboratory\\
    Charité University, Berlin, Germany\\
    Topic: Reentrant processing and visual awareness: neural and perceptual mechanisms 


    \vspace{3mm}
    % Education
    \section{\mysidestyle Education \\ \& Academic \\ Experience }
	
	\textbf{Post-doctoral Associate} in Thomas Serre’s laboratory \hfill \textbf{2010-2012}\\
	CLPS, Brown University, Providence, RI, USA\\
	Topic: Linking behavioral and electrophysiological data to computational models

    \textbf{Ph.D. in Neurosciences}, Université de Toulouse, CNRS, France \hfill \textbf{2010}\\
	Advisor:  Dr Simon J. Thorpe\\
	Topic: Ultra-rapid recognition of objects in natural scenes.\\
	Highest academic distinction: Très honorable avec les félicitations du jury à l'unanimité.\\
	Defense date: 12 July 2010
	\vspace{-1.5mm}

	\textbf{European Summer School in Visual Neurosciences} \hfill \textbf{Sep 2008}\\
	\textsl{'From Spike to Awareness'}, Organisation: K. Gegenfurtner, F. Bremmer, J. Braun.\\
	Rauischholzhausen, Germany
	\vspace{-1.5mm}

	\textbf{M.S. in Cognitive Science}, \textsl{Graduated magna cum laude} \hfill \textbf{2006}\\
	ENS / EHESS / Ecole Polytechnique / Paris 5 / Paris 6, France
	\vspace{-1.5mm}

	\textbf{B.S. in Cognitive Science}, \textsl{Graduated magna cum laude} \hfill \textbf{2004}\\
	Université Bordeaux 2, France

	%\textbf{DEUG in Psychology} \hfill \textbf{2003}\\
	%Université Paris 5, France

	%\textbf{Baccalauréat} \hfill \textbf{2000}\\
	%Lycée Bernard Palissy, Saintes, France

%\vspace{3mm}
%% Research Interests
% \section{\mysidestyle Research\\Interests}
% blabla


	\vspace{3mm}	
	% Publications
    \section{\mysidestyle Publications}

	\textbf{Refereed Journal Articles}\\\\
	Cauchoix* M, \textbf{Crouzet* SM}, Fize D \& Serre T (soon submitted). Early ventral visual stream activity enables rapid visual categorization.
	
	\vspace{-2mm} Wu* CT, \textbf{Crouzet* SM}, Thorpe SJ \& Fabre-Thorpe M (soon submitted). At 120 ms you know where the animal is but you don’t yet know it’s a dog.
		
	\vspace{-2mm} Cauchoix M \& \textbf{Crouzet SM} (submitted). How plausible is a subcortical account of rapid visual recognition?
	
	\vspace{-2mm} \textbf{Crouzet SM}, Joubert OR, Thorpe SJ, Fabre-Thorpe M (2012) Animal Detection Precedes Access to Scene Category. PLoS ONE 7(12): e51471. doi:10.1371/journal.pone.0051471

	\vspace{-2mm} \textbf{Crouzet SM} and Thorpe SJ (2011). Low level cues and ultra-fast face detection. Front. Psychology 2:342. doi: 10.3389/fpsyg.2011.00342

	\vspace{-2mm} \textbf{Crouzet SM} and Serre T (2011). What are the visual features underlying rapid object recognition? Front. Psychology 2:326. doi: 10.3389/fpsyg.2011.00326

	\vspace{-2mm} \textbf{Crouzet, S. M.}, Cauchoix, M. (2011). When does the visual system need to look back?  \textit{The Journal of Neuroscience}, 15 June 2011, 31(24): 8706-8707

	\vspace{-2mm} \textbf{Crouzet, S. M.}, Kirchner, H., \& Thorpe, S. J.  (2010). Fast saccades toward faces: Face detection in just 100 ms. \textit{Journal of Vision}, 10(4):16, 1-17, http://journalofvision.org/10/4/16/, doi:10.1167/10.4.16.

\newpage

	\vspace{3mm}	
	% Book chapters
	\textbf{Book Chapter}\\\\
	M., Fabre-Thorpe, \textbf{S. Crouzet}, G. A. Rousselet, H. Kirchner and S. J. Thorpe (2008). Catégorisation visuelle rapide: les visages sont-ils des 	objets spécifiques? In \textsl{Traitement et reconnaissance des visages: du percept \`a la personne}. E. J. Barbeau, S. Joubert and O. Felician. Marseille, Solal: 239-260.


\vspace{3mm}	
% Conference Presentations
    \section{\mysidestyle Conference\\Presentations}

\footnotesize % boring list so make it smaller

Crouzet SM, Hviid Del Pin S, Overgaard M, Busch NA (2013) Dynamics of saccadic responses reveal how object substitution masking interferes with reentrant processing. 55th TeaP - Tagung experimentell arbeitender Psychologen (Conference of Experimental Psychologists).

\vspace{-2mm} Crouzet SM, Cauchoix M, Fize D, Serre T (2011) The neural basis of rapid categorization: Linking computational models and electrophysiology. NIPS 2011 workshop on machine learning and interpretation in neuroimaging.

\vspace{-2mm} Cauchoix M., Crouzet S., Fize D., Serre T. (2011) Visual features and dynamics of rapid recognition in monkey visual cortex. SFN 2011

\vspace{-2mm} Crouzet S M, Stemmler T, Capps M, Fahle M \& Serre T (2011) Single-trial decoding of binocular rivalry switches from oculometric and pupil data. Vision Science Society, Naples, Florida.

\vspace{-2mm} Brilhault A, Mathey M, Jolmes N, Crouzet S M \& Thorpe SJ (2011) Saccades to color: an ultra-fast controllable mechanism to low-level features. Vision Science Society, Naples, Florida.

\vspace{-2mm} Thorpe S J, Brilhault A, Mathey M, Crouzet S M, 2010, "Colour based target selection for ultrarapid saccades: The fastest controllable selection mechanism?" Perception 39 ECVP Abstract Supplement, page 158

\vspace{-2mm} Mathey M A, Crouzet S M, Thorpe S J, 2010, "The accuracy of ultra-rapid saccades to faces" Perception 39 ECVP Abstract Supplement, page 171

\vspace{-2mm} Crouzet, S. M. \& Thorpe, S. J. (2010) Power spectrum cues underlying ultra-fast saccades towards faces [Abstract]. Journal of Vision, 10(7): 634

\vspace{-2mm} Mathey, M. A., Crouzet, S. M. \& Thorpe, S. J. (2010) Ultra-rapid saccades to faces : the effect of target size [Abstract]. Journal of Vision, 10(7): 635

\vspace{-2mm} Crouzet S, Mathey M, Thorpe S J (2009). Ultra-fast saccades to faces: A temporal precedence effect? Perception 38 ECVP Abstract Supplement, page 157.

\vspace{-2mm} Crouzet, S. M., Joubert, O. R., Thorpe, S. J., \& Fabre-Thorpe, M. (2009). The bear before the forest, but the city before the cars: Revealing early object/background processing [Abstract]. Journal of Vision, 9(8):954

\vspace{-2mm} Fabre-Thorpe, M., Crouzet, S. M., Wu, C.-T., \& Thorpe, S. J. (2009). At 130 ms you "know" where the animal is but you don't yet "know" it's a dog [Abstract]. Journal of Vision, 9(8):786

\vspace{-2mm} Thorpe, S. J., Crouzet, S. M., Macé, M. J., Bacon-Macé, N., \& Fabre-Thorpe, M. (2009). Masking in a high-level gender discrimination task is essentially entirely pre-cortical [Abstract]. Journal of Vision, 9(8):546

\vspace{-2mm} S Crouzet, H Kirchner, S J Thorpe (2008). Saccading towards faces in 100 ms. What's the secret? Perception 37 ECVP Abstract Supplement, page 119. 

\vspace{-2mm} S J Thorpe, H Kirchner, S Crouzet, P Bayerl, H Neumann (2008). Processing times for optic flow patterns measured by the saccadic choice task. Perception 37 ECVP Abstract Supplement, page 40.

\vspace{-2mm} Crouzet, S., Thorpe, S. J., \& Kirchner, H. (2007). Category-dependent variations in visual processing time. Journal of Vision, 7(9):922,922a, http://journalofvision.org/7/9/922/, doi:10.1167/7.9.922.

\vspace{-2mm} Thorpe, S., Crouzet, S., \& Kirchner, H. (2007). Saliency maps and ultra-rapid choice saccade tasks. Journal of Vision, 7(9):30, 30a, http://journalofvision.org/7/9/30/, doi:10.1167/7.9.30.

\vspace{-2mm} Simon J. Thorpe, Sébastien Crouzet, Holle Kirchner and Michèle Fabre-Thorpe (2006). Ultra-rapid face detection in natural images : implications for computation in the visual system. First French Conference on Computational Neurosciences, pp. 124-127. Abbaye des Prémontrés, Pont à Mousson, France.

\vspace{-2mm} Simon J. Thorpe, Sébastien Crouzet and Holle Kirchner (2006). Comparing processing speed for complex natural scenes and simple visual forms. Perception, vol. 35, p 128.

\normalsize


\newpage

% Invited talks
    \section{\mysidestyle Invited\\Talks}
    
\footnotesize % boring list so make it smaller

\textit{Invited by David Sheinberg}, Brown University, Providence, RI, USA \hfill \textbf{Mar 2012}\\
An early cortical basis for speed of sight. 

\vspace{-2mm} 
\textit{Invited by Simon J. Thorpe}, CERCO-CNRS, Toulouse, France \hfill \textbf{Jan 2012}\\
Rapid Visual Processing of Natural Scenes: Linking Behavioral and Electrophysiological Data to Computational Models.

\vspace{-2mm} 
\textit{In-House Seminar}, Neuroscience Department, Brown University, Providence, RI, USA \hfill \textbf{Nov 2011}\\
Rapid Visual Processing of Natural Scenes: Linking Behavioral and Electrophysiological Data to Computational Models.

\vspace{-2mm} 
\textit{Invited by Aude Oliva}, MIT, Cambridge, MA, USA \hfill \textbf{May 2009}\\
Revealing early visual processing of natural scenes using a saccade choice task.

\normalsize

\vspace{3mm}


   \section{\mysidestyle Peer-review\\Activities}

\vspace{3mm}
\begin{table}[ht]
\begin{tabular}{l r}
\textbf{Journal} & \textbf{\# articles reviewed} \\ \hline
Animal Cognition & 2\\
Frontiers in Perception Science (review editor) & 2\\
Attention, Perception, \& Psychophysics & 1\\
Brain Topography & 1\\
Cerebral Cortex & 1\\
IEEE Transactions on Pattern Analysis and Machine Intelligence & 1\\
PLOS ONE & 1\\
Psychological Science & 1\\
Seeing and Perceiving & 1\\
\end{tabular}
\end{table}

\vspace{3mm}

% Teaching
    \section{\mysidestyle Teaching\\Experience}
    
    \textbf{Tutorial at the Mind \& Brain Institute} \hfill \textbf{Dec 2012}\\
	\textsl{Mind \& Brain Institute, Berlin, Germany}\\
	R for data analysis

	\textbf{Guest lecture} \hfill \textbf{2011}\\
	\textsl{Computational Vision course, CLPS1520, Brown University, Providence, RI, USA}\\
	Object recognition in natural scenes

	\textbf{Teaching Assistant} \hfill \textbf{2006 to 2009}\\
	\textsl{Department of Psychology, Université Toulouse Le Mirail, Toulouse, France}\\
	Introduction to Neurosciences

	\textbf{Instructor} \hfill \textbf{2006 to 2009}\\
	\textsl{School of Psychomotricity, Faculté de Médecine de Rangueil, Toulouse, France}\\
	The visual system 

	\textbf{Instructor} \hfill \textbf{2006 to 2007}\\
	\textsl{School of Psychomotricity, Faculté de Médecine de Rangueil, Toulouse, France}\\
	Epistemology of neuropsychology

	\textbf{Instructor} \hfill \textbf{2006}\\
	\textsl{School of Psychomotricity, Faculté de Médecine de Rangueil, Toulouse, France}\\
	Sleep, emotions


\newpage

%\vspace{3mm}

% Teaching
%    \section{\mysidestyle Student\\Supervision}
%
%    Rohan Katipally - Undergraduate student\\
%    Michelle Evans - Undergraduate student\\
%    Marie Mathey - Master student\\
%   % Carla - Undergraduate\\

\vspace{3mm}
 % Official commitments
    \section{\mysidestyle Official\\ Commitments} % ou Responsibilities/Associations

	\textbf{Organizer of the J3CN} \hfill \textbf{2010 to 2011}\\
	\textsl{Journal Club for Cognitive \& Computational Neuroscience, Brown University}\\
	 \url{https://sites.google.com/a/brown.edu/j3cn/}\\
	Providence, USA	

	\textbf{Header of the Organizing Committee for the CJCSC'09} \hfill \textbf{2008 to 2009}\\
	\textsl{French Cognitive Science Young Researcher Conference}\\
	 \url{http://fresco.risc.cnrs.fr/cjcsc2009/}\\
	Toulouse, France		

	\textbf{Header of the Young Researcher Workshop for trend forecasting} \hfill \textbf{2009}\\
	 \textbf{in Cognitive Science}\\
	\textsl{Part of the PIRSTEC project funded by the French National Research Agency (ANR)}\\
	 \url{http://pirstec.risc.cnrs.fr}

	\textbf{Students and Post-Docs representative} \hfill \textbf{2006 to 2009}\\
	\textsl{Brain and Cognition Research Center lab council}\\
	Toulouse, France	

	\textbf{Founding member of the association inCOGnu} \hfill \textbf{2006 to 2009}\\
	\textsl{Association of cognitive science students of Toulouse}\\
	 \url{http://incognu.fr/}\\
	Toulouse, France



\vspace{3mm}
% Fellowships
    \section{\mysidestyle Fellowships, Grants \& Scholarships}

	\textbf{4th year of Ph.D. fellowship} \hfill \textbf{Nov 2009 to May 2010}\\
	\textsl{Fondation pour la Recherche Médicale (FRM)}

	\textbf{Postgraduate scholarship} \hfill \textbf{Oct 2006 to Sep 2009}\\ % ~ 100 000 Euros
	\textsl{Délégation Générale pour l'Armement (DGA, French Ministry of Defense)}

	\textbf{Master scholarship} \hfill \textbf{2005 to 2006}\\
	\textsl{René Descartes University (Paris 5)}


\vspace{3mm}
    \section{\mysidestyle Professional\\Societies}
	Society for Neuroscience\\
	Society of Visual Science
	
	
\vspace{3mm}
    \section{\mysidestyle Languages}
	\textsl{French}: Mother tongue\\
	\textsl{English}: Fluent\\
	\textsl{German}: Currently learning\\
	\textsl{Spanish}: Very elementary

\vspace{3mm}
% Technical Skills
    \section{\mysidestyle Professional\\Skills}

	\textsl{Operating Systems}: Advanced knowledge of Mac OS and GNU Linux.\\
	\textsl{Programming languages}: MATLAB, R, Python.\\
	\textsl{Experimental testing}: Psychtoolbox for MATLAB.\\
	\textsl{Eye movement recording}: SR Research Eyelink, SMI View Eyetracker, Chronos Eyetracker, EOG.\\
	\textsl{EEG and iEEG analysis}: Homemade MATLAB functions and EEGlab.\\
	\textsl{Statistical Analysis}: Parametric and non-parametric tests, Multivariate Pattern Analysis.\\
	\textsl{Communication and publishing}: Advanced knowledge of \LaTeX, Adobe Illustrator \& the presentation software Keynote (Mac OS); website creation and maintenance with HTML and CSS.


%\vspace{3mm}
\newpage	

%% Referees
	\section{\mysidestyle Referees} 

	\begin{tabular}{@{}p{6cm}p{6cm}}
	\textbf{Dr Simon J. Thorpe}       &  \textbf{Dr Thomas Serre}                   \\
	Ph.D. advisor                               &  Post-doc advisor                       \\
	CNRS, Toulouse, France          &  Brown University, Providence, RI, USA        \\
	phone: \textsl{available on request}    &  phone: \textsl{available on request}     \\
	e-mail: \textsl{\href{mailto:simon.thorpe@cerco.ups-tlse.fr}{simon.thorpe@cerco.ups-tlse.fr}}   &  
	e-mail: \textsl{\href{mailto:thomas_serre@brown.edu}{thomas\_serre@brown.edu}}    \\
	\end{tabular}
	
	\textbf{Dr Niko A. Busch} \\
	Post-doc advisor\\
	Charité University, Berlin, Germany \\
	phone: \textsl{available on request} \\
	e-mail: \textsl{\href{mailto:niko.busch@charite.de}{niko.busch@charite.de}} \\

\end{resume}
\end{document}